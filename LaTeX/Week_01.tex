\documentclass[a4paper]{article}
\usepackage{amsmath}
\usepackage{enumerate}
\usepackage[RPvoltages, american]{circuitikz}
\begin{document}
  \subsection*{Problem 1}
  \subsubsection*{A}
  \begin{enumerate}[i.]
    \item $11_{2}:$
    $1\times 2^{1}+1\times 2^{0} = 3$
    
    \item $101_{2}:$
    $1\times 2^{2}+0\times 2^{1}+1\times 2{0} = 5$
    
    \item $1000_{2}:$
    $1\times 2^{3}+0\times 2^{2}+0\times 2^{1}+0\times 2^{0} = 8$
    
    \item $1101_{2}:$
    $2^{3}+2^{2}+0^{3}+2^{0} = 13$
    
    \item $101011_{2}:$
    $2^{5}+0^{4}+2^{3}+0^{2}+2^{1}+2^{0} = 43$
    
    \item $100110_{2}:$
    $2^{5}+0^{4}+0^{3}+2^{2}+2^{1}+0^{0} = 38$
     
    \item $11110_{2}:$
    $2^{4}+2^{3}+2^{2}+2^{1}+0^{0} = 30$
    
    \item $10000000_{2}:$
    $2^{7}+0^{6}+0^{5}+0^{4}+0^{3}+0^{2}+0^{1}+0^{0} = 128$
    
    \item $11111111_{2}:$
    $2^{7}+2^{6}+2^{5}+2^{4}+2^{3}+2^{2}+2^{1}+2^{0} = 255$
    
    \item $11010111_{2}:$
    $2^{7}+2^{6}+0^{5}+2^{4}+0^{3}+2^{2}+2^{1}+2^{0} = 215$
    
  \end{enumerate}
  \subsubsection*{B}

  \begin{enumerate}[i.]
    \item $7_{10} = 111_{2}$
    \item $10_{10} = 1010_{2} $
    \item $33_{10} = 100001_{2}$
    \item $50_{10} = 110010_{2}$
    \item $96_{10} = 1100000_{2}$
    \item $108_{10} = 1101100_{2}$
    \item $214_{10} = 11010110_{2}$
    \item $15_{10} = 1111_{2}$
    \item $71_{10} = 1000111_{2}$
    \item $146_{10} = 10010010_{2}$
  \end{enumerate}

  \subsubsection*{C}
  \begin{enumerate}
    \item \begin{tabular}{c@{\,}c@{\,}c@{\,}c}
      & & 1 \\
    + & & 1 \\
    \hline
    & 1 & 0
    \end{tabular}
    \item \begin{tabular}{c@{\,}c@{\,}c@{\,}c}
      & 1 & 1 \\
      + & 1 & 0 \\
      \hline
      1 & 0 & 1
    \end{tabular}
    \item \begin{tabular}{c@{\,}c@{\,}c@{\,}c@{\,}c}
      & 1 & 1 & 0 \\
      + & 0 & 1 & 1 \\
      \hline
      1 & 0 & 0 & 1
    \end{tabular}
    \item \begin{tabular}{c@{\,}c@{\,}c@{\,}c@{\,}c@{\,}c}
      & 1 & 1 & 0 & 1 \\
      + & 1 & 0 & 1 & 1 \\
      \hline
      1 & 1 & 0 & 0 & 0
    \end{tabular}
    \item \begin{tabular}{c@{\,}c@{\,}c@{\,}c@{\,}c@{\,}c}
      & 0 & 0 & 1 & 0 & 1 \\
      + & 1 & 1 & 1 & 1 & 0 \\
      \hline
      1 & 0 & 0 & 0 & 1 & 1 
    \end{tabular}
  \end{enumerate}
  \subsection*{Problem 2}
  \begin{enumerate}[a)]

    \item \begin {displaymath} 
      \begin{array}{|c|c|c|}
        \hline
        A_{in} & B_{in} & A \land B\\
        \hline
        0 & 0 & 0\\
        0 & 1 & 0\\
        1 & 0 & 0\\
        1 & 1 & 1\\
        \hline
      \end{array}  
    \end{displaymath}
    % Graphic for TeX using PGF
% Title: /home/afoek/Diagram1.dia
% Creator: Dia v0.97+git
% CreationDate: Wed Oct 28 10:44:29 2020
% For: afoek
% \usepackage{tikz}
% The following commands are not supported in PSTricks at present
% We define them conditionally, so when they are implemented,
% this pgf file will use them.
\ifx\du\undefined
  \newlength{\du}
\fi
\setlength{\du}{15\unitlength}
\begin{tikzpicture}[even odd rule]
\pgftransformxscale{1.000000}
\pgftransformyscale{-1.000000}
\definecolor{dialinecolor}{rgb}{0.000000, 0.000000, 0.000000}
\pgfsetstrokecolor{dialinecolor}
\pgfsetstrokeopacity{1.000000}
\definecolor{diafillcolor}{rgb}{1.000000, 1.000000, 1.000000}
\pgfsetfillcolor{diafillcolor}
\pgfsetfillopacity{1.000000}
\pgfsetlinewidth{0.100000\du}
\pgfsetdash{}{0pt}
\pgfsetbuttcap
\pgfsetmiterjoin
\pgfsetlinewidth{0.100000\du}
\pgfsetbuttcap
\pgfsetmiterjoin
\pgfsetdash{}{0pt}
\definecolor{diafillcolor}{rgb}{1.000000, 1.000000, 1.000000}
\pgfsetfillcolor{diafillcolor}
\pgfsetfillopacity{1.000000}
\definecolor{dialinecolor}{rgb}{0.000000, 0.000000, 0.000000}
\pgfsetstrokecolor{dialinecolor}
\pgfsetstrokeopacity{1.000000}
\pgfpathmoveto{\pgfpoint{-11.982500\du}{3.082446\du}}
\pgfpathcurveto{\pgfpoint{-9.828308\du}{3.082446\du}}{\pgfpoint{-9.828308\du}{5.963666\du}}{\pgfpoint{-9.828308\du}{7.884479\du}}
\pgfpathcurveto{\pgfpoint{-10.905404\du}{7.884479\du}}{\pgfpoint{-13.059596\du}{7.884479\du}}{\pgfpoint{-14.136692\du}{7.884479\du}}
\pgfpathcurveto{\pgfpoint{-14.136692\du}{5.963666\du}}{\pgfpoint{-14.136692\du}{3.082446\du}}{\pgfpoint{-11.982500\du}{3.082446\du}}
\pgfpathclose
\pgfusepath{fill,stroke}
\pgfsetlinewidth{0.010000\du}
\pgfsetbuttcap
\pgfsetmiterjoin
\pgfsetdash{}{0pt}
\definecolor{dialinecolor}{rgb}{0.000000, 0.000000, 0.000000}
\pgfsetstrokecolor{dialinecolor}
\pgfsetstrokeopacity{1.000000}
\pgfpathmoveto{\pgfpoint{-11.982500\du}{3.082446\du}}
\pgfpathcurveto{\pgfpoint{-9.828308\du}{3.082446\du}}{\pgfpoint{-9.828308\du}{5.963666\du}}{\pgfpoint{-9.828308\du}{7.884479\du}}
\pgfpathcurveto{\pgfpoint{-10.905404\du}{7.884479\du}}{\pgfpoint{-13.059596\du}{7.884479\du}}{\pgfpoint{-14.136692\du}{7.884479\du}}
\pgfpathcurveto{\pgfpoint{-14.136692\du}{5.963666\du}}{\pgfpoint{-14.136692\du}{3.082446\du}}{\pgfpoint{-11.982500\du}{3.082446\du}}
\pgfusepath{stroke}
\pgfsetlinewidth{0.100000\du}
\pgfsetdash{}{0pt}
\pgfsetbuttcap
\definecolor{dialinecolor}{rgb}{0.000000, 0.000000, 0.000000}
\pgfsetstrokecolor{dialinecolor}
\pgfsetstrokeopacity{1.000000}
\draw (-13.059596\du,7.884479\du)--(-13.063368\du,9.591405\du);
\pgfsetlinewidth{0.100000\du}
\pgfsetdash{}{0pt}
\pgfsetbuttcap
\definecolor{dialinecolor}{rgb}{0.000000, 0.000000, 0.000000}
\pgfsetstrokecolor{dialinecolor}
\pgfsetstrokeopacity{1.000000}
\draw (-10.905404\du,7.884479\du)--(-10.904279\du,9.625783\du);
% setfont left to latex
\definecolor{dialinecolor}{rgb}{0.000000, 0.000000, 0.000000}
\pgfsetstrokecolor{dialinecolor}
\pgfsetstrokeopacity{1.000000}
\definecolor{diafillcolor}{rgb}{0.000000, 0.000000, 0.000000}
\pgfsetfillcolor{diafillcolor}
\pgfsetfillopacity{1.000000}
\node[anchor=base west,inner sep=0pt,outer sep=0pt,color=dialinecolor] at (-13.897814\du,10.346991\du){Ain};
% setfont left to latex
\definecolor{dialinecolor}{rgb}{0.000000, 0.000000, 0.000000}
\pgfsetstrokecolor{dialinecolor}
\pgfsetstrokeopacity{1.000000}
\definecolor{diafillcolor}{rgb}{0.000000, 0.000000, 0.000000}
\pgfsetfillcolor{diafillcolor}
\pgfsetfillopacity{1.000000}
\node[anchor=base west,inner sep=0pt,outer sep=0pt,color=dialinecolor] at (-11.784556\du,10.384771\du){Bin};
\pgfsetlinewidth{0.100000\du}
\pgfsetdash{}{0pt}
\pgfsetbuttcap
\definecolor{dialinecolor}{rgb}{0.000000, 0.000000, 0.000000}
\pgfsetstrokecolor{dialinecolor}
\pgfsetstrokeopacity{1.000000}
\draw (-11.982500\du,3.082446\du)--(-11.983572\du,1.233239\du);
\pgfsetdash{}{0pt}
\pgfsetmiterjoin
\pgfsetbuttcap
\definecolor{diafillcolor}{rgb}{0.000000, 0.000000, 0.000000}
\pgfsetfillcolor{diafillcolor}
\pgfsetfillopacity{1.000000}
\fill (-11.983766\du,0.898866\du)--(-11.733507\du,1.344582\du)--(-11.983572\du,1.233239\du)--(-12.233507\du,1.344812\du)--cycle;
\definecolor{dialinecolor}{rgb}{0.000000, 0.000000, 0.000000}
\pgfsetstrokecolor{dialinecolor}
\pgfsetstrokeopacity{1.000000}
\draw (-11.983766\du,0.898866\du)--(-11.733507\du,1.344582\du)--(-11.983572\du,1.233239\du)--(-12.233507\du,1.344812\du)--cycle;
% setfont left to latex
\definecolor{dialinecolor}{rgb}{0.000000, 0.000000, 0.000000}
\pgfsetstrokecolor{dialinecolor}
\pgfsetstrokeopacity{1.000000}
\definecolor{diafillcolor}{rgb}{0.000000, 0.000000, 0.000000}
\pgfsetfillcolor{diafillcolor}
\pgfsetfillopacity{1.000000}
\node[anchor=base west,inner sep=0pt,outer sep=0pt,color=dialinecolor] at (-12.626410\du,0.724689\du){OUT};
\end{tikzpicture}

    
    \item \begin {displaymath} 
      \begin{array}{|c|c|c|}
        \hline
        A_{in} & B_{in} & A \oplus B\\
        \hline
        0 & 0 & 0\\
        0 & 1 & 1\\
        1 & 0 & 1\\
        1 & 1 & 0\\
        \hline
      \end{array}
    \end{displaymath}
    % Graphic for TeX using PGF
% Title: /home/afoek/Diagram1.dia
% Creator: Dia v0.97+git
% CreationDate: Wed Oct 28 10:47:15 2020
% For: afoek
% \usepackage{tikz}
% The following commands are not supported in PSTricks at present
% We define them conditionally, so when they are implemented,
% this pgf file will use them.
\ifx\du\undefined
  \newlength{\du}
\fi
\setlength{\du}{15\unitlength}
\begin{tikzpicture}[even odd rule]
\pgftransformxscale{1.000000}
\pgftransformyscale{-1.000000}
\definecolor{dialinecolor}{rgb}{0.000000, 0.000000, 0.000000}
\pgfsetstrokecolor{dialinecolor}
\pgfsetstrokeopacity{1.000000}
\definecolor{diafillcolor}{rgb}{1.000000, 1.000000, 1.000000}
\pgfsetfillcolor{diafillcolor}
\pgfsetfillopacity{1.000000}
\pgfsetlinewidth{0.100000\du}
\pgfsetdash{}{0pt}
\pgfsetbuttcap
{
\definecolor{diafillcolor}{rgb}{0.000000, 0.000000, 0.000000}
\pgfsetfillcolor{diafillcolor}
\pgfsetfillopacity{1.000000}
% was here!!!
\definecolor{dialinecolor}{rgb}{0.000000, 0.000000, 0.000000}
\pgfsetstrokecolor{dialinecolor}
\pgfsetstrokeopacity{1.000000}
\draw (-12.801296\du,8.181200\du)--(-12.805069\du,10.095516\du);
}
\pgfsetlinewidth{0.100000\du}
\pgfsetdash{}{0pt}
\pgfsetbuttcap
{
\definecolor{diafillcolor}{rgb}{0.000000, 0.000000, 0.000000}
\pgfsetfillcolor{diafillcolor}
\pgfsetfillopacity{1.000000}
% was here!!!
\definecolor{dialinecolor}{rgb}{0.000000, 0.000000, 0.000000}
\pgfsetstrokecolor{dialinecolor}
\pgfsetstrokeopacity{1.000000}
\draw (-11.132507\du,8.214988\du)--(-11.156455\du,10.106795\du);
}
% setfont left to latex
\definecolor{dialinecolor}{rgb}{0.000000, 0.000000, 0.000000}
\pgfsetstrokecolor{dialinecolor}
\pgfsetstrokeopacity{1.000000}
\definecolor{diafillcolor}{rgb}{0.000000, 0.000000, 0.000000}
\pgfsetfillcolor{diafillcolor}
\pgfsetfillopacity{1.000000}
\node[anchor=base west,inner sep=0pt,outer sep=0pt,color=dialinecolor] at (-13.714413\du,11.007330\du){Ain};
% setfont left to latex
\definecolor{dialinecolor}{rgb}{0.000000, 0.000000, 0.000000}
\pgfsetstrokecolor{dialinecolor}
\pgfsetstrokeopacity{1.000000}
\definecolor{diafillcolor}{rgb}{0.000000, 0.000000, 0.000000}
\pgfsetfillcolor{diafillcolor}
\pgfsetfillopacity{1.000000}
\node[anchor=base west,inner sep=0pt,outer sep=0pt,color=dialinecolor] at (-11.853331\du,11.003850\du){Bin};
\pgfsetlinewidth{0.100000\du}
\pgfsetdash{}{0pt}
\pgfsetbuttcap
{
\definecolor{diafillcolor}{rgb}{0.000000, 0.000000, 0.000000}
\pgfsetfillcolor{diafillcolor}
\pgfsetfillopacity{1.000000}
% was here!!!
\pgfsetarrowsend{stealth}
\definecolor{dialinecolor}{rgb}{0.000000, 0.000000, 0.000000}
\pgfsetstrokecolor{dialinecolor}
\pgfsetstrokeopacity{1.000000}
\draw (-11.982500\du,3.433270\du)--(-11.983824\du,0.872583\du);
}
% setfont left to latex
\definecolor{dialinecolor}{rgb}{0.000000, 0.000000, 0.000000}
\pgfsetstrokecolor{dialinecolor}
\pgfsetstrokeopacity{1.000000}
\definecolor{diafillcolor}{rgb}{0.000000, 0.000000, 0.000000}
\pgfsetfillcolor{diafillcolor}
\pgfsetfillopacity{1.000000}
\node[anchor=base west,inner sep=0pt,outer sep=0pt,color=dialinecolor] at (-12.626410\du,0.789046\du){OUT};
\pgfsetlinewidth{0.100000\du}
\pgfsetdash{}{0pt}
\pgfsetbuttcap
\pgfsetmiterjoin
\pgfsetlinewidth{0.100000\du}
\pgfsetbuttcap
\pgfsetmiterjoin
\pgfsetdash{}{0pt}
\definecolor{diafillcolor}{rgb}{1.000000, 1.000000, 1.000000}
\pgfsetfillcolor{diafillcolor}
\pgfsetfillopacity{1.000000}
\definecolor{dialinecolor}{rgb}{0.000000, 0.000000, 0.000000}
\pgfsetstrokecolor{dialinecolor}
\pgfsetstrokeopacity{1.000000}
\pgfpathmoveto{\pgfpoint{-12.011674\du}{3.467488\du}}
\pgfpathcurveto{\pgfpoint{-10.253341\du}{4.346654\du}}{\pgfpoint{-10.253341\du}{6.104988\du}}{\pgfpoint{-10.253341\du}{7.863321\du}}
\pgfpathcurveto{\pgfpoint{-11.132507\du}{6.984154\du}}{\pgfpoint{-12.890841\du}{6.984154\du}}{\pgfpoint{-13.770007\du}{7.863321\du}}
\pgfpathcurveto{\pgfpoint{-13.770007\du}{6.104988\du}}{\pgfpoint{-13.770007\du}{4.346654\du}}{\pgfpoint{-12.011674\du}{3.467488\du}}
\pgfpathclose
\pgfusepath{fill,stroke}
\pgfsetlinewidth{0.010000\du}
\pgfsetbuttcap
\pgfsetmiterjoin
\pgfsetdash{}{0pt}
\definecolor{dialinecolor}{rgb}{0.000000, 0.000000, 0.000000}
\pgfsetstrokecolor{dialinecolor}
\pgfsetstrokeopacity{1.000000}
\pgfpathmoveto{\pgfpoint{-12.011674\du}{3.467488\du}}
\pgfpathcurveto{\pgfpoint{-10.253341\du}{4.346654\du}}{\pgfpoint{-10.253341\du}{6.104988\du}}{\pgfpoint{-10.253341\du}{7.863321\du}}
\pgfpathcurveto{\pgfpoint{-11.132507\du}{6.984154\du}}{\pgfpoint{-12.890841\du}{6.984154\du}}{\pgfpoint{-13.770007\du}{7.863321\du}}
\pgfpathcurveto{\pgfpoint{-13.770007\du}{6.104988\du}}{\pgfpoint{-13.770007\du}{4.346654\du}}{\pgfpoint{-12.011674\du}{3.467488\du}}
\pgfusepath{stroke}
\pgfsetlinewidth{0.100000\du}
\pgfsetbuttcap
\pgfsetmiterjoin
\pgfsetdash{}{0pt}
\definecolor{dialinecolor}{rgb}{0.000000, 0.000000, 0.000000}
\pgfsetstrokecolor{dialinecolor}
\pgfsetstrokeopacity{1.000000}
\pgfpathmoveto{\pgfpoint{-13.770007\du}{8.742488\du}}
\pgfpathcurveto{\pgfpoint{-12.890841\du}{7.863321\du}}{\pgfpoint{-11.132507\du}{7.863321\du}}{\pgfpoint{-10.253341\du}{8.742488\du}}
\pgfusepath{stroke}
\end{tikzpicture}

    
    \item \begin {displaymath}
      \begin{array}{|c|c|c|c|}
        \hline
        A_{in} & B_{in} & A \lor B & \neg(A \lor B)\\
        \hline
        0 & 0 & 0 & 1\\
        0 & 1 & 1 & 0\\
        1 & 0 & 1 & 0\\
        1 & 1 & 1 & 0\\
        \hline
      \end{array}  
    \end{displaymath}
    % Graphic for TeX using PGF
% Title: /home/afoek/Diagram1.dia
% Creator: Dia v0.97+git
% CreationDate: Wed Oct 28 11:04:17 2020
% For: afoek
% \usepackage{tikz}
% The following commands are not supported in PSTricks at present
% We define them conditionally, so when they are implemented,
% this pgf file will use them.
\ifx\du\undefined
  \newlength{\du}
\fi
\setlength{\du}{15\unitlength}
\begin{tikzpicture}[even odd rule]
\pgftransformxscale{1.000000}
\pgftransformyscale{-1.000000}
\definecolor{dialinecolor}{rgb}{0.000000, 0.000000, 0.000000}
\pgfsetstrokecolor{dialinecolor}
\pgfsetstrokeopacity{1.000000}
\definecolor{diafillcolor}{rgb}{1.000000, 1.000000, 1.000000}
\pgfsetfillcolor{diafillcolor}
\pgfsetfillopacity{1.000000}
\pgfsetlinewidth{0.100000\du}
\pgfsetdash{}{0pt}
\pgfsetbuttcap
{
\definecolor{diafillcolor}{rgb}{0.000000, 0.000000, 0.000000}
\pgfsetfillcolor{diafillcolor}
\pgfsetfillopacity{1.000000}
% was here!!!
\definecolor{dialinecolor}{rgb}{0.000000, 0.000000, 0.000000}
\pgfsetstrokecolor{dialinecolor}
\pgfsetstrokeopacity{1.000000}
\draw (-13.079938\du,15.130063\du)--(-13.081358\du,17.054512\du);
}
\pgfsetlinewidth{0.100000\du}
\pgfsetdash{}{0pt}
\pgfsetbuttcap
{
\definecolor{diafillcolor}{rgb}{0.000000, 0.000000, 0.000000}
\pgfsetfillcolor{diafillcolor}
\pgfsetfillopacity{1.000000}
% was here!!!
\definecolor{dialinecolor}{rgb}{0.000000, 0.000000, 0.000000}
\pgfsetstrokecolor{dialinecolor}
\pgfsetstrokeopacity{1.000000}
\draw (-11.223647\du,15.130063\du)--(-11.239922\du,17.046206\du);
}
% setfont left to latex
\definecolor{dialinecolor}{rgb}{0.000000, 0.000000, 0.000000}
\pgfsetstrokecolor{dialinecolor}
\pgfsetstrokeopacity{1.000000}
\definecolor{diafillcolor}{rgb}{0.000000, 0.000000, 0.000000}
\pgfsetfillcolor{diafillcolor}
\pgfsetfillopacity{1.000000}
\node[anchor=base west,inner sep=0pt,outer sep=0pt,color=dialinecolor] at (-13.892715\du,17.956571\du){Ain};
% setfont left to latex
\definecolor{dialinecolor}{rgb}{0.000000, 0.000000, 0.000000}
\pgfsetstrokecolor{dialinecolor}
\pgfsetstrokeopacity{1.000000}
\definecolor{diafillcolor}{rgb}{0.000000, 0.000000, 0.000000}
\pgfsetfillcolor{diafillcolor}
\pgfsetfillopacity{1.000000}
\node[anchor=base west,inner sep=0pt,outer sep=0pt,color=dialinecolor] at (-12.031633\du,18.006124\du){Bin};
\pgfsetlinewidth{0.100000\du}
\pgfsetdash{}{0pt}
\pgfsetbuttcap
{
\definecolor{diafillcolor}{rgb}{0.000000, 0.000000, 0.000000}
\pgfsetfillcolor{diafillcolor}
\pgfsetfillopacity{1.000000}
% was here!!!
\pgfsetarrowsend{stealth}
\definecolor{dialinecolor}{rgb}{0.000000, 0.000000, 0.000000}
\pgfsetstrokecolor{dialinecolor}
\pgfsetstrokeopacity{1.000000}
\draw (-12.151793\du,11.046223\du)--(-12.151286\du,8.434567\du);
}
% setfont left to latex
\definecolor{dialinecolor}{rgb}{0.000000, 0.000000, 0.000000}
\pgfsetstrokecolor{dialinecolor}
\pgfsetstrokeopacity{1.000000}
\definecolor{diafillcolor}{rgb}{0.000000, 0.000000, 0.000000}
\pgfsetfillcolor{diafillcolor}
\pgfsetfillopacity{1.000000}
\node[anchor=base west,inner sep=0pt,outer sep=0pt,color=dialinecolor] at (-13.122908\du,8.463069\du){OUT 1};
\pgfsetlinewidth{0.100000\du}
\pgfsetdash{}{0pt}
\pgfsetbuttcap
\pgfsetmiterjoin
\pgfsetlinewidth{0.100000\du}
\pgfsetbuttcap
\pgfsetmiterjoin
\pgfsetdash{}{0pt}
\definecolor{diafillcolor}{rgb}{1.000000, 1.000000, 1.000000}
\pgfsetfillcolor{diafillcolor}
\pgfsetfillopacity{1.000000}
\definecolor{dialinecolor}{rgb}{0.000000, 0.000000, 0.000000}
\pgfsetstrokecolor{dialinecolor}
\pgfsetstrokeopacity{1.000000}
\pgfpathmoveto{\pgfpoint{-12.151793\du}{11.046223\du}}
\pgfpathcurveto{\pgfpoint{-10.295502\du}{11.974368\du}}{\pgfpoint{-10.295502\du}{13.830659\du}}{\pgfpoint{-10.295502\du}{15.686950\du}}
\pgfpathcurveto{\pgfpoint{-11.223647\du}{14.758804\du}}{\pgfpoint{-13.079938\du}{14.758804\du}}{\pgfpoint{-14.008083\du}{15.686950\du}}
\pgfpathcurveto{\pgfpoint{-14.008083\du}{13.830659\du}}{\pgfpoint{-14.008083\du}{11.974368\du}}{\pgfpoint{-12.151793\du}{11.046223\du}}
\pgfpathclose
\pgfusepath{fill,stroke}
\pgfsetlinewidth{0.010000\du}
\pgfsetbuttcap
\pgfsetmiterjoin
\pgfsetdash{}{0pt}
\definecolor{dialinecolor}{rgb}{0.000000, 0.000000, 0.000000}
\pgfsetstrokecolor{dialinecolor}
\pgfsetstrokeopacity{1.000000}
\pgfpathmoveto{\pgfpoint{-12.151793\du}{11.046223\du}}
\pgfpathcurveto{\pgfpoint{-10.295502\du}{11.974368\du}}{\pgfpoint{-10.295502\du}{13.830659\du}}{\pgfpoint{-10.295502\du}{15.686950\du}}
\pgfpathcurveto{\pgfpoint{-11.223647\du}{14.758804\du}}{\pgfpoint{-13.079938\du}{14.758804\du}}{\pgfpoint{-14.008083\du}{15.686950\du}}
\pgfpathcurveto{\pgfpoint{-14.008083\du}{13.830659\du}}{\pgfpoint{-14.008083\du}{11.974368\du}}{\pgfpoint{-12.151793\du}{11.046223\du}}
\pgfusepath{stroke}
\pgfsetlinewidth{0.100000\du}
\pgfsetdash{}{0pt}
\pgfsetbuttcap
\pgfsetmiterjoin
\pgfsetlinewidth{0.100000\du}
\pgfsetbuttcap
\pgfsetmiterjoin
\pgfsetdash{}{0pt}
\definecolor{diafillcolor}{rgb}{1.000000, 1.000000, 1.000000}
\pgfsetfillcolor{diafillcolor}
\pgfsetfillopacity{1.000000}
\fill (-12.157043\du,3.694533\du)--(-11.004886\du,5.614794\du)--(-13.309200\du,5.614794\du)--cycle;
\definecolor{dialinecolor}{rgb}{0.000000, 0.000000, 0.000000}
\pgfsetstrokecolor{dialinecolor}
\pgfsetstrokeopacity{1.000000}
\draw (-12.157043\du,3.694533\du)--(-11.004886\du,5.614794\du)--(-13.309200\du,5.614794\du)--cycle;
\pgfsetlinewidth{0.010000\du}
\pgfsetbuttcap
\pgfsetmiterjoin
\pgfsetdash{}{0pt}
\definecolor{dialinecolor}{rgb}{0.000000, 0.000000, 0.000000}
\pgfsetstrokecolor{dialinecolor}
\pgfsetstrokeopacity{1.000000}
\draw (-12.157043\du,3.694533\du)--(-11.004886\du,5.614794\du)--(-13.309200\du,5.614794\du)--cycle;
\pgfsetlinewidth{0.100000\du}
\pgfsetbuttcap
\pgfsetmiterjoin
\pgfsetdash{}{0pt}
\definecolor{diafillcolor}{rgb}{1.000000, 1.000000, 1.000000}
\pgfsetfillcolor{diafillcolor}
\pgfsetfillopacity{1.000000}
\pgfpathellipse{\pgfpoint{-12.157043\du}{3.310480\du}}{\pgfpoint{0.384052\du}{0\du}}{\pgfpoint{0\du}{0.384052\du}}
\pgfusepath{fill}
\definecolor{dialinecolor}{rgb}{0.000000, 0.000000, 0.000000}
\pgfsetstrokecolor{dialinecolor}
\pgfsetstrokeopacity{1.000000}
\pgfpathellipse{\pgfpoint{-12.157043\du}{3.310480\du}}{\pgfpoint{0.384052\du}{0\du}}{\pgfpoint{0\du}{0.384052\du}}
\pgfusepath{stroke}
\pgfsetlinewidth{0.010000\du}
\pgfsetbuttcap
\pgfsetmiterjoin
\pgfsetdash{}{0pt}
\definecolor{dialinecolor}{rgb}{0.000000, 0.000000, 0.000000}
\pgfsetstrokecolor{dialinecolor}
\pgfsetstrokeopacity{1.000000}
\pgfpathellipse{\pgfpoint{-12.157043\du}{3.310480\du}}{\pgfpoint{0.384052\du}{0\du}}{\pgfpoint{0\du}{0.384052\du}}
\pgfusepath{stroke}
\pgfsetlinewidth{0.100000\du}
\pgfsetdash{}{0pt}
\pgfsetbuttcap
{
\definecolor{diafillcolor}{rgb}{0.000000, 0.000000, 0.000000}
\pgfsetfillcolor{diafillcolor}
\pgfsetfillopacity{1.000000}
% was here!!!
\pgfsetarrowsend{stealth}
\definecolor{dialinecolor}{rgb}{0.000000, 0.000000, 0.000000}
\pgfsetstrokecolor{dialinecolor}
\pgfsetstrokeopacity{1.000000}
\draw (-12.156345\du,7.889211\du)--(-12.157043\du,5.614794\du);
}
\pgfsetlinewidth{0.100000\du}
\pgfsetdash{}{0pt}
\pgfsetbuttcap
{
\definecolor{diafillcolor}{rgb}{0.000000, 0.000000, 0.000000}
\pgfsetfillcolor{diafillcolor}
\pgfsetfillopacity{1.000000}
% was here!!!
\pgfsetarrowsend{stealth}
\definecolor{dialinecolor}{rgb}{0.000000, 0.000000, 0.000000}
\pgfsetstrokecolor{dialinecolor}
\pgfsetstrokeopacity{1.000000}
\draw (-12.157043\du,2.926428\du)--(-12.136421\du,1.143289\du);
}
% setfont left to latex
\definecolor{dialinecolor}{rgb}{0.000000, 0.000000, 0.000000}
\pgfsetstrokecolor{dialinecolor}
\pgfsetstrokeopacity{1.000000}
\definecolor{diafillcolor}{rgb}{0.000000, 0.000000, 0.000000}
\pgfsetfillcolor{diafillcolor}
\pgfsetfillopacity{1.000000}
\node[anchor=base west,inner sep=0pt,outer sep=0pt,color=dialinecolor] at (-13.147240\du,1.054099\du){OUT 2};
\end{tikzpicture}

    
    \item \begin {displaymath}
      \begin{array}{|c|c|c|c|}
        \hline
        A_{in} & B_{in} & \neg B & A \oplus \neg B\\
        \hline
        0 & 0 & 0 & 1\\
        0 & 1 & 1 & 0\\
        1 & 0 & 1 & 0\\
        1 & 1 & 1 & 1\\
        \hline
      \end{array}
    \end{displaymath}
    % Graphic for TeX using PGF
% Title: /home/afoek/Diagram1.dia
% Creator: Dia v0.97+git
% CreationDate: Wed Oct 28 11:16:40 2020
% For: afoek
% \usepackage{tikz}
% The following commands are not supported in PSTricks at present
% We define them conditionally, so when they are implemented,
% this pgf file will use them.
\ifx\du\undefined
  \newlength{\du}
\fi
\setlength{\du}{15\unitlength}
\begin{tikzpicture}[even odd rule]
\pgftransformxscale{1.000000}
\pgftransformyscale{-1.000000}
\definecolor{dialinecolor}{rgb}{0.000000, 0.000000, 0.000000}
\pgfsetstrokecolor{dialinecolor}
\pgfsetstrokeopacity{1.000000}
\definecolor{diafillcolor}{rgb}{1.000000, 1.000000, 1.000000}
\pgfsetfillcolor{diafillcolor}
\pgfsetfillopacity{1.000000}
\pgfsetlinewidth{0.100000\du}
\pgfsetdash{}{0pt}
\pgfsetbuttcap
\pgfsetmiterjoin
\pgfsetlinewidth{0.100000\du}
\pgfsetbuttcap
\pgfsetmiterjoin
\pgfsetdash{}{0pt}
\definecolor{diafillcolor}{rgb}{1.000000, 1.000000, 1.000000}
\pgfsetfillcolor{diafillcolor}
\pgfsetfillopacity{1.000000}
\definecolor{dialinecolor}{rgb}{0.000000, 0.000000, 0.000000}
\pgfsetstrokecolor{dialinecolor}
\pgfsetstrokeopacity{1.000000}
\pgfpathmoveto{\pgfpoint{-11.484259\du}{3.128100\du}}
\pgfpathcurveto{\pgfpoint{-9.853920\du}{3.943270\du}}{\pgfpoint{-9.853920\du}{5.573608\du}}{\pgfpoint{-9.853920\du}{7.203946\du}}
\pgfpathcurveto{\pgfpoint{-10.669090\du}{6.388777\du}}{\pgfpoint{-12.299428\du}{6.388777\du}}{\pgfpoint{-13.114597\du}{7.203946\du}}
\pgfpathcurveto{\pgfpoint{-13.114597\du}{5.573608\du}}{\pgfpoint{-13.114597\du}{3.943270\du}}{\pgfpoint{-11.484259\du}{3.128100\du}}
\pgfpathclose
\pgfusepath{fill,stroke}
\pgfsetlinewidth{0.010000\du}
\pgfsetbuttcap
\pgfsetmiterjoin
\pgfsetdash{}{0pt}
\definecolor{dialinecolor}{rgb}{0.000000, 0.000000, 0.000000}
\pgfsetstrokecolor{dialinecolor}
\pgfsetstrokeopacity{1.000000}
\pgfpathmoveto{\pgfpoint{-11.484259\du}{3.128100\du}}
\pgfpathcurveto{\pgfpoint{-9.853920\du}{3.943270\du}}{\pgfpoint{-9.853920\du}{5.573608\du}}{\pgfpoint{-9.853920\du}{7.203946\du}}
\pgfpathcurveto{\pgfpoint{-10.669090\du}{6.388777\du}}{\pgfpoint{-12.299428\du}{6.388777\du}}{\pgfpoint{-13.114597\du}{7.203946\du}}
\pgfpathcurveto{\pgfpoint{-13.114597\du}{5.573608\du}}{\pgfpoint{-13.114597\du}{3.943270\du}}{\pgfpoint{-11.484259\du}{3.128100\du}}
\pgfusepath{stroke}
\pgfsetlinewidth{0.100000\du}
\pgfsetbuttcap
\pgfsetmiterjoin
\pgfsetdash{}{0pt}
\definecolor{dialinecolor}{rgb}{0.000000, 0.000000, 0.000000}
\pgfsetstrokecolor{dialinecolor}
\pgfsetstrokeopacity{1.000000}
\pgfpathmoveto{\pgfpoint{-13.114597\du}{8.019116\du}}
\pgfpathcurveto{\pgfpoint{-12.299428\du}{7.203946\du}}{\pgfpoint{-10.669090\du}{7.203946\du}}{\pgfpoint{-9.853920\du}{8.019116\du}}
\pgfusepath{stroke}
\pgfsetlinewidth{0.100000\du}
\pgfsetdash{}{0pt}
\pgfsetbuttcap
\pgfsetmiterjoin
\pgfsetlinewidth{0.100000\du}
\pgfsetbuttcap
\pgfsetmiterjoin
\pgfsetdash{}{0pt}
\definecolor{diafillcolor}{rgb}{1.000000, 1.000000, 1.000000}
\pgfsetfillcolor{diafillcolor}
\pgfsetfillopacity{1.000000}
\fill (-10.654066\du,9.876582\du)--(-9.547779\du,11.720392\du)--(-11.760352\du,11.720392\du)--cycle;
\definecolor{dialinecolor}{rgb}{0.000000, 0.000000, 0.000000}
\pgfsetstrokecolor{dialinecolor}
\pgfsetstrokeopacity{1.000000}
\draw (-10.654066\du,9.876582\du)--(-9.547779\du,11.720392\du)--(-11.760352\du,11.720392\du)--cycle;
\pgfsetlinewidth{0.010000\du}
\pgfsetbuttcap
\pgfsetmiterjoin
\pgfsetdash{}{0pt}
\definecolor{dialinecolor}{rgb}{0.000000, 0.000000, 0.000000}
\pgfsetstrokecolor{dialinecolor}
\pgfsetstrokeopacity{1.000000}
\draw (-10.654066\du,9.876582\du)--(-9.547779\du,11.720392\du)--(-11.760352\du,11.720392\du)--cycle;
\pgfsetlinewidth{0.100000\du}
\pgfsetbuttcap
\pgfsetmiterjoin
\pgfsetdash{}{0pt}
\definecolor{diafillcolor}{rgb}{1.000000, 1.000000, 1.000000}
\pgfsetfillcolor{diafillcolor}
\pgfsetfillopacity{1.000000}
\pgfpathellipse{\pgfpoint{-10.654066\du}{9.507820\du}}{\pgfpoint{0.368762\du}{0\du}}{\pgfpoint{0\du}{0.368762\du}}
\pgfusepath{fill}
\definecolor{dialinecolor}{rgb}{0.000000, 0.000000, 0.000000}
\pgfsetstrokecolor{dialinecolor}
\pgfsetstrokeopacity{1.000000}
\pgfpathellipse{\pgfpoint{-10.654066\du}{9.507820\du}}{\pgfpoint{0.368762\du}{0\du}}{\pgfpoint{0\du}{0.368762\du}}
\pgfusepath{stroke}
\pgfsetlinewidth{0.010000\du}
\pgfsetbuttcap
\pgfsetmiterjoin
\pgfsetdash{}{0pt}
\definecolor{dialinecolor}{rgb}{0.000000, 0.000000, 0.000000}
\pgfsetstrokecolor{dialinecolor}
\pgfsetstrokeopacity{1.000000}
\pgfpathellipse{\pgfpoint{-10.654066\du}{9.507820\du}}{\pgfpoint{0.368762\du}{0\du}}{\pgfpoint{0\du}{0.368762\du}}
\pgfusepath{stroke}
\pgfsetlinewidth{0.100000\du}
\pgfsetdash{}{0pt}
\pgfsetbuttcap
{
\definecolor{diafillcolor}{rgb}{0.000000, 0.000000, 0.000000}
\pgfsetfillcolor{diafillcolor}
\pgfsetfillopacity{1.000000}
% was here!!!
\pgfsetarrowsend{stealth}
\definecolor{dialinecolor}{rgb}{0.000000, 0.000000, 0.000000}
\pgfsetstrokecolor{dialinecolor}
\pgfsetstrokeopacity{1.000000}
\draw (-10.654066\du,9.139057\du)--(-10.669090\du,7.530014\du);
}
\pgfsetlinewidth{0.100000\du}
\pgfsetdash{}{0pt}
\pgfsetbuttcap
{
\definecolor{diafillcolor}{rgb}{0.000000, 0.000000, 0.000000}
\pgfsetfillcolor{diafillcolor}
\pgfsetfillopacity{1.000000}
% was here!!!
\pgfsetarrowsend{stealth}
\definecolor{dialinecolor}{rgb}{0.000000, 0.000000, 0.000000}
\pgfsetstrokecolor{dialinecolor}
\pgfsetstrokeopacity{1.000000}
\draw (-12.285520\du,13.922910\du)--(-12.299428\du,7.530014\du);
}
% setfont left to latex
\definecolor{dialinecolor}{rgb}{0.000000, 0.000000, 0.000000}
\pgfsetstrokecolor{dialinecolor}
\pgfsetstrokeopacity{1.000000}
\definecolor{diafillcolor}{rgb}{0.000000, 0.000000, 0.000000}
\pgfsetfillcolor{diafillcolor}
\pgfsetfillopacity{1.000000}
\node[anchor=base west,inner sep=0pt,outer sep=0pt,color=dialinecolor] at (-12.526506\du,14.518741\du){A};
\pgfsetlinewidth{0.100000\du}
\pgfsetdash{}{0pt}
\pgfsetbuttcap
{
\definecolor{diafillcolor}{rgb}{0.000000, 0.000000, 0.000000}
\pgfsetfillcolor{diafillcolor}
\pgfsetfillopacity{1.000000}
% was here!!!
\pgfsetarrowsend{stealth}
\definecolor{dialinecolor}{rgb}{0.000000, 0.000000, 0.000000}
\pgfsetstrokecolor{dialinecolor}
\pgfsetstrokeopacity{1.000000}
\draw (-10.687832\du,13.985977\du)--(-10.654066\du,11.720392\du);
}
% setfont left to latex
\definecolor{dialinecolor}{rgb}{0.000000, 0.000000, 0.000000}
\pgfsetstrokecolor{dialinecolor}
\pgfsetstrokeopacity{1.000000}
\definecolor{diafillcolor}{rgb}{0.000000, 0.000000, 0.000000}
\pgfsetfillcolor{diafillcolor}
\pgfsetfillopacity{1.000000}
\node[anchor=base west,inner sep=0pt,outer sep=0pt,color=dialinecolor] at (-10.830087\du,14.536188\du){B};
\pgfsetlinewidth{0.100000\du}
\pgfsetdash{}{0pt}
\pgfsetbuttcap
{
\definecolor{diafillcolor}{rgb}{0.000000, 0.000000, 0.000000}
\pgfsetfillcolor{diafillcolor}
\pgfsetfillopacity{1.000000}
% was here!!!
\pgfsetarrowsend{stealth}
\definecolor{dialinecolor}{rgb}{0.000000, 0.000000, 0.000000}
\pgfsetstrokecolor{dialinecolor}
\pgfsetstrokeopacity{1.000000}
\draw (-11.484259\du,3.128100\du)--(-11.486676\du,0.820818\du);
}
% setfont left to latex
\definecolor{dialinecolor}{rgb}{0.000000, 0.000000, 0.000000}
\pgfsetstrokecolor{dialinecolor}
\pgfsetstrokeopacity{1.000000}
\definecolor{diafillcolor}{rgb}{0.000000, 0.000000, 0.000000}
\pgfsetfillcolor{diafillcolor}
\pgfsetfillopacity{1.000000}
\node[anchor=base west,inner sep=0pt,outer sep=0pt,color=dialinecolor] at (-12.117342\du,0.694685\du){OUT};
\end{tikzpicture}

    
    \item \begin {displaymath}
      \begin{array}{|c|c|c|c|c|}
        \hline
        A_{in} & B_{in} & C_{in} & A \lor B & (A \lor B)\oplus C\\
        \hline
        0 & 0 & 0 & 0 & 0\\
        0 & 0 & 1 & 0 & 1\\
        0 & 1 & 0 & 1 & 1\\
        0 & 1 & 1 & 1 & 0\\
        1 & 0 & 0 & 1 & 1\\
        1 & 0 & 1 & 1 & 0\\
        1 & 1 & 0 & 1 & 1\\
        1 & 1 & 1 & 1 & 0\\
        \hline
      \end{array}
    \end{displaymath}
    % Graphic for TeX using PGF
% Title: /home/afoek/Diagram1.dia
% Creator: Dia v0.97+git
% CreationDate: Wed Oct 28 11:34:55 2020
% For: afoek
% \usepackage{tikz}
% The following commands are not supported in PSTricks at present
% We define them conditionally, so when they are implemented,
% this pgf file will use them.
\ifx\du\undefined
  \newlength{\du}
\fi
\setlength{\du}{15\unitlength}
\begin{tikzpicture}[even odd rule]
\pgftransformxscale{1.000000}
\pgftransformyscale{-1.000000}
\definecolor{dialinecolor}{rgb}{0.000000, 0.000000, 0.000000}
\pgfsetstrokecolor{dialinecolor}
\pgfsetstrokeopacity{1.000000}
\definecolor{diafillcolor}{rgb}{1.000000, 1.000000, 1.000000}
\pgfsetfillcolor{diafillcolor}
\pgfsetfillopacity{1.000000}
\pgfsetlinewidth{0.100000\du}
\pgfsetdash{}{0pt}
\pgfsetbuttcap
\pgfsetmiterjoin
\pgfsetlinewidth{0.100000\du}
\pgfsetbuttcap
\pgfsetmiterjoin
\pgfsetdash{}{0pt}
\definecolor{diafillcolor}{rgb}{1.000000, 1.000000, 1.000000}
\pgfsetfillcolor{diafillcolor}
\pgfsetfillopacity{1.000000}
\definecolor{dialinecolor}{rgb}{0.000000, 0.000000, 0.000000}
\pgfsetstrokecolor{dialinecolor}
\pgfsetstrokeopacity{1.000000}
\pgfpathmoveto{\pgfpoint{-11.275178\du}{10.134061\du}}
\pgfpathcurveto{\pgfpoint{-9.885373\du}{10.828963\du}}{\pgfpoint{-9.885373\du}{12.218767\du}}{\pgfpoint{-9.885373\du}{13.608572\du}}
\pgfpathcurveto{\pgfpoint{-10.580276\du}{12.913670\du}}{\pgfpoint{-11.970080\du}{12.913670\du}}{\pgfpoint{-12.664982\du}{13.608572\du}}
\pgfpathcurveto{\pgfpoint{-12.664982\du}{12.218767\du}}{\pgfpoint{-12.664982\du}{10.828963\du}}{\pgfpoint{-11.275178\du}{10.134061\du}}
\pgfpathclose
\pgfusepath{fill,stroke}
\pgfsetlinewidth{0.010000\du}
\pgfsetbuttcap
\pgfsetmiterjoin
\pgfsetdash{}{0pt}
\definecolor{dialinecolor}{rgb}{0.000000, 0.000000, 0.000000}
\pgfsetstrokecolor{dialinecolor}
\pgfsetstrokeopacity{1.000000}
\pgfpathmoveto{\pgfpoint{-11.275178\du}{10.134061\du}}
\pgfpathcurveto{\pgfpoint{-9.885373\du}{10.828963\du}}{\pgfpoint{-9.885373\du}{12.218767\du}}{\pgfpoint{-9.885373\du}{13.608572\du}}
\pgfpathcurveto{\pgfpoint{-10.580276\du}{12.913670\du}}{\pgfpoint{-11.970080\du}{12.913670\du}}{\pgfpoint{-12.664982\du}{13.608572\du}}
\pgfpathcurveto{\pgfpoint{-12.664982\du}{12.218767\du}}{\pgfpoint{-12.664982\du}{10.828963\du}}{\pgfpoint{-11.275178\du}{10.134061\du}}
\pgfusepath{stroke}
\pgfsetlinewidth{0.100000\du}
\pgfsetdash{}{0pt}
\pgfsetbuttcap
{
\definecolor{diafillcolor}{rgb}{0.000000, 0.000000, 0.000000}
\pgfsetfillcolor{diafillcolor}
\pgfsetfillopacity{1.000000}
% was here!!!
\pgfsetarrowsend{stealth}
\definecolor{dialinecolor}{rgb}{0.000000, 0.000000, 0.000000}
\pgfsetstrokecolor{dialinecolor}
\pgfsetstrokeopacity{1.000000}
\draw (-11.977160\du,15.926808\du)--(-11.970080\du,13.191630\du);
}
\pgfsetlinewidth{0.100000\du}
\pgfsetdash{}{0pt}
\pgfsetbuttcap
{
\definecolor{diafillcolor}{rgb}{0.000000, 0.000000, 0.000000}
\pgfsetfillcolor{diafillcolor}
\pgfsetfillopacity{1.000000}
% was here!!!
\pgfsetarrowsend{stealth}
\definecolor{dialinecolor}{rgb}{0.000000, 0.000000, 0.000000}
\pgfsetstrokecolor{dialinecolor}
\pgfsetstrokeopacity{1.000000}
\draw (-10.583070\du,15.926808\du)--(-10.580276\du,13.191630\du);
}
\pgfsetlinewidth{0.100000\du}
\pgfsetdash{}{0pt}
\pgfsetbuttcap
{
\definecolor{diafillcolor}{rgb}{0.000000, 0.000000, 0.000000}
\pgfsetfillcolor{diafillcolor}
\pgfsetfillopacity{1.000000}
% was here!!!
\pgfsetarrowsend{stealth}
\definecolor{dialinecolor}{rgb}{0.000000, 0.000000, 0.000000}
\pgfsetstrokecolor{dialinecolor}
\pgfsetstrokeopacity{1.000000}
\draw (-11.275178\du,10.134061\du)--(-11.302628\du,6.985362\du);
}
\pgfsetlinewidth{0.100000\du}
\pgfsetdash{}{0pt}
\pgfsetbuttcap
\pgfsetmiterjoin
\pgfsetlinewidth{0.100000\du}
\pgfsetbuttcap
\pgfsetmiterjoin
\pgfsetdash{}{0pt}
\definecolor{diafillcolor}{rgb}{1.000000, 1.000000, 1.000000}
\pgfsetfillcolor{diafillcolor}
\pgfsetfillopacity{1.000000}
\definecolor{dialinecolor}{rgb}{0.000000, 0.000000, 0.000000}
\pgfsetstrokecolor{dialinecolor}
\pgfsetstrokeopacity{1.000000}
\pgfpathmoveto{\pgfpoint{-10.607833\du}{3.233471\du}}
\pgfpathcurveto{\pgfpoint{-9.218244\du}{3.928266\du}}{\pgfpoint{-9.218244\du}{5.317855\du}}{\pgfpoint{-9.218244\du}{6.707444\du}}
\pgfpathcurveto{\pgfpoint{-9.913038\du}{6.012650\du}}{\pgfpoint{-11.302628\du}{6.012650\du}}{\pgfpoint{-11.997422\du}{6.707444\du}}
\pgfpathcurveto{\pgfpoint{-11.997422\du}{5.317855\du}}{\pgfpoint{-11.997422\du}{3.928266\du}}{\pgfpoint{-10.607833\du}{3.233471\du}}
\pgfpathclose
\pgfusepath{fill,stroke}
\pgfsetlinewidth{0.010000\du}
\pgfsetbuttcap
\pgfsetmiterjoin
\pgfsetdash{}{0pt}
\definecolor{dialinecolor}{rgb}{0.000000, 0.000000, 0.000000}
\pgfsetstrokecolor{dialinecolor}
\pgfsetstrokeopacity{1.000000}
\pgfpathmoveto{\pgfpoint{-10.607833\du}{3.233471\du}}
\pgfpathcurveto{\pgfpoint{-9.218244\du}{3.928266\du}}{\pgfpoint{-9.218244\du}{5.317855\du}}{\pgfpoint{-9.218244\du}{6.707444\du}}
\pgfpathcurveto{\pgfpoint{-9.913038\du}{6.012650\du}}{\pgfpoint{-11.302628\du}{6.012650\du}}{\pgfpoint{-11.997422\du}{6.707444\du}}
\pgfpathcurveto{\pgfpoint{-11.997422\du}{5.317855\du}}{\pgfpoint{-11.997422\du}{3.928266\du}}{\pgfpoint{-10.607833\du}{3.233471\du}}
\pgfusepath{stroke}
\pgfsetlinewidth{0.100000\du}
\pgfsetbuttcap
\pgfsetmiterjoin
\pgfsetdash{}{0pt}
\definecolor{dialinecolor}{rgb}{0.000000, 0.000000, 0.000000}
\pgfsetstrokecolor{dialinecolor}
\pgfsetstrokeopacity{1.000000}
\pgfpathmoveto{\pgfpoint{-11.997422\du}{7.402239\du}}
\pgfpathcurveto{\pgfpoint{-11.302628\du}{6.707444\du}}{\pgfpoint{-9.913038\du}{6.707444\du}}{\pgfpoint{-9.218244\du}{7.402239\du}}
\pgfusepath{stroke}
\pgfsetlinewidth{0.100000\du}
\pgfsetdash{}{0pt}
\pgfsetbuttcap
{
\definecolor{diafillcolor}{rgb}{0.000000, 0.000000, 0.000000}
\pgfsetfillcolor{diafillcolor}
\pgfsetfillopacity{1.000000}
% was here!!!
\pgfsetarrowsend{stealth}
\definecolor{dialinecolor}{rgb}{0.000000, 0.000000, 0.000000}
\pgfsetstrokecolor{dialinecolor}
\pgfsetstrokeopacity{1.000000}
\draw (-9.946570\du,9.262711\du)--(-9.913038\du,6.985362\du);
}
% setfont left to latex
\definecolor{dialinecolor}{rgb}{0.000000, 0.000000, 0.000000}
\pgfsetstrokecolor{dialinecolor}
\pgfsetstrokeopacity{1.000000}
\definecolor{diafillcolor}{rgb}{0.000000, 0.000000, 0.000000}
\pgfsetfillcolor{diafillcolor}
\pgfsetfillopacity{1.000000}
\node[anchor=base west,inner sep=0pt,outer sep=0pt,color=dialinecolor] at (-12.193206\du,16.495505\du){A};
% setfont left to latex
\definecolor{dialinecolor}{rgb}{0.000000, 0.000000, 0.000000}
\pgfsetstrokecolor{dialinecolor}
\pgfsetstrokeopacity{1.000000}
\definecolor{diafillcolor}{rgb}{0.000000, 0.000000, 0.000000}
\pgfsetfillcolor{diafillcolor}
\pgfsetfillopacity{1.000000}
\node[anchor=base west,inner sep=0pt,outer sep=0pt,color=dialinecolor] at (-10.771865\du,16.472580\du){B};
% setfont left to latex
\definecolor{dialinecolor}{rgb}{0.000000, 0.000000, 0.000000}
\pgfsetstrokecolor{dialinecolor}
\pgfsetstrokeopacity{1.000000}
\definecolor{diafillcolor}{rgb}{0.000000, 0.000000, 0.000000}
\pgfsetfillcolor{diafillcolor}
\pgfsetfillopacity{1.000000}
\node[anchor=base west,inner sep=0pt,outer sep=0pt,color=dialinecolor] at (-10.175818\du,9.881682\du){C};
\pgfsetlinewidth{0.100000\du}
\pgfsetdash{}{0pt}
\pgfsetbuttcap
{
\definecolor{diafillcolor}{rgb}{0.000000, 0.000000, 0.000000}
\pgfsetfillcolor{diafillcolor}
\pgfsetfillopacity{1.000000}
% was here!!!
\pgfsetarrowsend{stealth}
\definecolor{dialinecolor}{rgb}{0.000000, 0.000000, 0.000000}
\pgfsetstrokecolor{dialinecolor}
\pgfsetstrokeopacity{1.000000}
\draw (-10.607833\du,3.233471\du)--(-10.634315\du,1.112921\du);
}
% setfont left to latex
\definecolor{dialinecolor}{rgb}{0.000000, 0.000000, 0.000000}
\pgfsetstrokecolor{dialinecolor}
\pgfsetstrokeopacity{1.000000}
\definecolor{diafillcolor}{rgb}{0.000000, 0.000000, 0.000000}
\pgfsetfillcolor{diafillcolor}
\pgfsetfillopacity{1.000000}
\node[anchor=base west,inner sep=0pt,outer sep=0pt,color=dialinecolor] at (-11.253287\du,1.021222\du){OUT};
\end{tikzpicture}

  \end{enumerate}
  This document write using \LaTeX \ author: Felix Montalfu(03082180055)
\end{document}
